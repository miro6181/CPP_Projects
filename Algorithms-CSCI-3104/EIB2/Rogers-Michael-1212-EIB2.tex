\documentclass[12pt]{article}
\setlength{\oddsidemargin}{0in}
\setlength{\evensidemargin}{0in}
\setlength{\textwidth}{6.5in}
\setlength{\parindent}{0in}
\setlength{\parskip}{\baselineskip}
\usepackage{amsmath,amsfonts,amssymb}
\usepackage{graphicx}
\usepackage[]{algorithmicx}

\usepackage{fancyhdr}
\pagestyle{fancy}

%\usepackage{hyperref}


\setlength{\headsep}{36pt}

\begin{document}

\lhead{{\bf CSCI 3104, Algorithms \\ Explain-It-Back 2} }
\rhead{Name: \fbox{Michael Rogers} \\ ID: \fbox{105667404} \\ {\bf Profs.\ Grochow \& Layer\\ Spring 2019, CU-Boulder}}
\renewcommand{\headrulewidth}{0.5pt}

\phantom{Test}

At a recent seminar, you learn that your colleagues in the astronomy department
have gained access to new radio telescopes that will drastically increase the
amount of data they can collect. In this seminar, these scientists are excited
about how this data will significantly improve their predictive models. You
politely ask a question about their ability to process all of this new data,
and they assure you that they have made all of the necessary hardware
improvements to accommodate this new technology.

In a short email, help your colleagues understand that faster processors and
more hard drives cannot make up for inefficient algorithms and data structures.
Try and convince them of the value of asymptotic analysis, and how based on the
result of that analysis they may need to also improve their algorithms and data
structures.


\pagebreak

\newpage
\mbox{Hello Colleagues in the Astronomy Department,}
\\ \\ Thank you for taking the time to explain the way in which you process data with your current (and now updated) system. It is very exciting to hear about these new radio telescopes and how much insight they can provide. I wanted to follow up on my question about how you process the enormous amount of data that you now have. You said that you have made all the hardware upgrades necessary to handle this new wave of data, but I have some unfortunate news. While having efficient and upgraded hardware is great for the speed of the computer, the way in which you process the data at the software level is significantly more important to the success and efficiency of your program. Let me elaborate, more sophisticated hardware does make a computer run faster; but as the data set grows in a specific program, the effectiveness of the hardware becomes less significant while the method you are using to process the data becomes more significant. In the computer science field, we call this the asymptotic analysis of an algorithm. Asymptotic analysis is a method we use to measure the efficiency of a program. The easiest way to explain it is this: the more calculations that a program must execute, the slower it becomes, and its efficiency decreases exponentially with a larger data set. The inefficiency of the program becomes much more drastic with an inefficient algorithm than it does with lower tier hardware. It is imperative that a program has an efficient method to process all that data or all of the money you have spent on the new hardware will be for naught. There will also be the revision of your method after wasting time and money running the data through an insufficient algorithm. I would be happy to help you implement this method, feel free to email me if you have any questions.

Good Luck!
\\ A Computer Scientist you know

\newpage
\pagebreak
\end{document}


