\documentclass[12pt]{article}
\setlength{\oddsidemargin}{0in}
\setlength{\evensidemargin}{0in}
\setlength{\textwidth}{6.5in}
\setlength{\parindent}{0in}
\setlength{\parskip}{\baselineskip}

\usepackage{amsmath,amsfonts,amssymb}
\usepackage{graphicx}
\usepackage{fancyhdr}
\pagestyle{fancy}
\usepackage{xcolor}
\usepackage{hyperref}


\begin{document}

\lhead{{\bf CSCI 3104 \\ Problem Set 5} }
\rhead{Name: \fbox{Michael Rogers} \\ ID: \fbox{105667404} \\ {\bf Profs.\ Grochow \& Layer\\ Spring 2019, CU-Boulder}}
\renewcommand{\headrulewidth}{0.5pt}
\phantom{Test}

Quick links: \ref{1a} \ref{1b} $\qquad$ \ref{2a} \ref{2b} \ref{2c} $\qquad$ \ref{3a} \ref{3b} 
\vspace{-3mm}

\vspace{-3mm}
\begin{enumerate}
		 
	\item Harry Potter is writing a secret message to Hermione and wants to prevent it from being understood by Voldemort. He decides to use Huffman encoding to encode the message. Magically, the symbol frequencies of the message are given by \textit{generalized Fibonacci numbers}, a famous sequence of integers known since antiquity. The $n$th generalized Fibonacci number is defined by $F_{n} = F_{n-1}+F_{n-2}$ for $n>1$ with base cases $F_{0}=3$ and $F_{1}=4$.
	\begin{enumerate}
	\item (5 pts) \label{1a} For an alphabet of $\Sigma=\{a,b,c,d,e,f,g,h\}$ with frequencies given by the first $|\Sigma|$ non-zero generalized Fibonacci numbers, give an optimal Huffman code and the corresponding encoding tree for Harry to use.
    \\ \\ \\ \\I don't know
    \pagebreak	
    	
	\item (15 pts) \label{1b} Generalize your answer to (\ref{1a}) and give the structure of an optimal code when there are $n$ letters, with the frequencies being the first $n$ non-zero generalized Fibonacci numbers. Prove your answer is correct.
    \\ \\ \\ \\I don't know
    \pagebreak	
    	
	\end{enumerate}



	\item A good hash function $h(x)$ behaves in practice very close to the uniform hashing assumption analyzed in class, but is a deterministic function. That is, $h(x)=k$ each time $x$ is used as an argument to $h()$. Designing good hash functions is hard, and a bad hash function can cause a hash table to quickly exit the sparse loading regime by overloading some buckets and under-loading others. Good hash functions often rely on beautiful and complicated insights from number theory, and have deep connections to pseudorandom number generators and cryptographic functions. In practice, most hash functions are moderate to poor approximations of uniform hashing.
	
	\smallskip Consider the following hash function. Let $U$ be the universe of strings composed of the characters from the alphabet $\Sigma=[${\tt A}$,\dots,${\tt Z}$]$, and let the function $f(x_{i})$ return the index of a letter $x_{i}\in \Sigma$, e.g., $f(${\tt A}$)=1$ and $f(${\tt Z}$)=26$. Finally, for an $m$-character string $x\in \Sigma^{m}$, define $h(x) = \left(\left[\sum_{i=1}^{m}f(x_{i})\right]\!\! \mod \ell\right)$, where $\ell$ is the number of buckets in the hash table. That is, our hash function sums up the index values of the characters of a string $x$ and maps that value onto one of the $\ell$ buckets.
	
	\begin{enumerate}
	\item \label{2a} (10 pts) The following list contains US Census derived last names: \\
	{\tt http://www2.census.gov/topics/genealogy/2000surnames/names.zip} \\
	(We have also provided a copy of a the CSV file from this ZIP file in the assignment on Moodle.) \\
	Using these names as input strings, first choose a uniformly random 50\% of these name strings and then hash them using $h(x)$.
	
	Produce a histogram showing the corresponding distribution of hash locations when $\ell=175$. Label the axes of your figure. Briefly describe what the figure shows about $h(x)$, and justify your results in terms of the behavior of $h(x)$. Do not forget to submit your code with your PS, using the same filename convention as for problem sets {\tt Lastname-Firstname-MMDD-PS5-code.*} (where the ``*'' can be any plaintext file type you like).
	 
	{\footnotesize Hint: the raw file includes information other than name strings, which will need to be removed; and, think about how you can count hash locations without building or using a real hash table.}
	\newpage
	I don't know
	\phantom{Delete this once you write your answer}
	\pagebreak
	
		THIS PAGE IS A PLACEHOLDER FOR CODE; DO NOT PUT ANYTHING ON THIS PAGE (EVEN CODE) IT'S JUST TO FOOL GRADESCOPE INTO WORKING CORRECTLY
	
	\pagebreak


	\item \label{2b} (5 pts) Enumerate at least 4 reasons why $h(x)$ is a bad hash function relative to the ideal behavior of uniform hashing.
	\\ \\ \\ \\I don't know
	\pagebreak
	
	
	\item \label{2c} (10 pts) Produce a plot showing (i) the length of the longest chain (were we to use chaining for resolving collisions under $h(x)$) as a function of the number $n$ of these strings that we hash into a table with $\ell=175$ buckets, (ii) the exact upper bound on the depth of a balanced binary tree with $n$ items stored, and (iii) the length of the longest chain were we to use a uniform hash instead of $h(x)$. Include a guide of $c\,n$, that is, graph $c n$ on the same chart, for some value of $c$ you find useful, so that you can compare this graph to that of your data.
	
	Then, comment on (i) how much shorter the longest chain would be under a uniform hash than under $h(x)$, and (ii)  the value of $n$ at which the balanced binary tree becomes a more efficient data structure than $h(x)$ and separately a uniform hash.
	\\ \\ \\ \\I don't know
	\pagebreak

	\end{enumerate}
	


	\item Draco Malfoy is struggling with the problem of making change for $n$ cents using the smallest number of coins. Malfoy has coin values of $v_{1} > v_{2} > \dots > v_{r}$ for $r$ coins types, where each coin's value $v_{i}$ is a positive integer. His goal is to obtain a set of counts $\{d_{i}\}$, one for each coin type, such that $\sum_{i=1}^{r}d_{i}=k$ and where $k$ is minimized.
	\begin{enumerate}
	\item \label{3a} (7 pts) A greedy algorithm for making change is the \textbf{cashier's algorithm}, which all young wizards learn. Malfoy writes the following pseudocode on the whiteboard to illustrate it, where $n$ is the amount of money to make change for and $v$ is a vector of the coin denominations:
	%
	\begin{small}
	\begin{verbatim}
	wizardChange(n,v,r) :
	   d[1 .. r] = 1       // initial histogram of coin types in solution
	   while n > 0 {
	      k = r
	      while ( k > 0 and v[k] > n ) { k++ }
	      if k==0 { return `no solution' }
	      else { n = n - v[k] }
	   }
	   return d
	\end{verbatim}
	\end{small}
	Hermione scoffs and says Malfoy's code has bugs. Identify the bugs and explain why each would cause the algorithm to fail.
	\\ \\ \\ \\I don't know
	\pagebreak
	

	\item \label{3b} (8 pts) Sometimes the goblins at Gringotts Wizarding Bank run out of coins,%
	%
	\footnote{It's a little known secret, but goblins like to \textit{eat} the coins. It isn't pretty for the coins, in the end.}
	%
	and make change using whatever is left on hand. Identify a set of Euro coin denominations (a subset of the denominations {\color{blue} \href{https://www.google.com/search?q=euro+coin+denominations}{\underline{here}}})\footnote{\href{https://www.google.com/search?q=euro+coin+denominations}{\tt https://www.google.com/search?q=euro+coin+denominations}} for which the greedy algorithm does not yield an optimal solution for making change. Justify your answer in terms of optimal substructure and the greedy-choice property. (The set should include the 1 Euro cent so that there is a solution for every value of $n$.) \\ \\
	
	I don't know
	
	
	\end{enumerate}

	
	

\end{enumerate}


\end{document}


