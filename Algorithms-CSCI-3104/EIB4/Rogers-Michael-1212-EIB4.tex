\documentclass[12pt]{article}
\setlength{\oddsidemargin}{0in}
\setlength{\evensidemargin}{0in}
\setlength{\textwidth}{6.5in}
\setlength{\parindent}{0in}
\setlength{\parskip}{\baselineskip}
\usepackage{amsmath,amsfonts,amssymb}
\usepackage{graphicx}
\usepackage[]{algorithmicx}

\usepackage{fancyhdr}
\pagestyle{fancy}

%\usepackage{hyperref}


\setlength{\headsep}{36pt}

\begin{document}

\lhead{{\bf CSCI 3104, Algorithms \\ Explain-It-Back 4} }
\rhead{Name: \fbox{Michael Rogers} \\ ID: \fbox{105667404} \\ {\bf Profs.\ Grochow \& Layer\\ Spring 2019, CU-Boulder}}
\renewcommand{\headrulewidth}{0.5pt}

\phantom{Test}

The ecology department is planning a large-scale fish migration study that
involves electronically tagging and releasing millions of fish across North
America, waiting six months, then trapping the fish and recording the where
they found each fish, according to its tracking number. In previous
smaller-scale experiments, the field scientists used a hand-held device that
had a sensor for reading the electronic sensor and a small onboard hard drive
that used a predefined table for storing the tag ID, timestamp, and current
GPS. In this table, every possible tag had a preset row which allowed for very
fast (constant-time) insertions and lookups. While the team would like to
re-use this hardware, they do not think that there is enough hard drive space
to account for a table with millions of rows. Help them figure out another
solution that provides fast insertions and lookups without requiring large
memory allocations. HINT: an individual scientist will only tag a few thousand
fish at a time.


\pagebreak

\newpage
\mbox{}
\newpage
\pagebreak
\end{document}


